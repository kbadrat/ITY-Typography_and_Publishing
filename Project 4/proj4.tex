\documentclass[a4paper, 11pt]{article}

\usepackage[left=2cm,top=3cm,text={17cm, 24cm}]{geometry}
\usepackage[czech]{babel}
\usepackage[utf8]{inputenc}
\usepackage{times}
\usepackage{bm}
\usepackage[IL2]{fontenc}
\usepackage[unicode]{hyperref}

\begin{document}

\begin{titlepage}
\begin{center}

\Huge \textsc{Vysoké učení technické v~Brně}\\
\huge\textsc{Fakulta informačních technologií} \\
\vspace{\stretch{0.382}}
\LARGE Typografie a publikování\,--\,3. projekt \\
\Huge {Bibliografické citace} \\
\vspace{\stretch{0.618}}

\end{center}

{\Large\today \hfill
Kovalets Vladyslav}

\end{titlepage}

\newpage

\section{Typografie}

\subsection{Co je typografie}

Typografie je nauka o~tom, jak má tiskovina vypadat, aby byla dobře čitelná, snadno se v~ní orientovalo a aby působila esteticky. Více informací naleznete v~\cite{Hrones2016}.

\subsection{Pravidla české digitální typografie}
České proto, že se liší od typografie zahraniční a digitální proto, že  ruční či strojový tisk byl vytlačen během několika prvních let nástupu počítačů. Pokud zveřejňujete jakékoli dokumenty prostřednictvím počítače, bude to pro vás užitečné \cite{Sirucek2007}.


Máme také pro vás typografický tahák \cite{Valkova2020}, který vysvětluje, jak některé znaky a symboly správně napsat.

\subsection{Vliv na nás}

Jaké je nejlepší písmo pro milostný dopis, Times nebo Comic Sans? To znáte? A~proč právě tímto písmem si můžete přečíst v~tomto článku \cite{Pilka2019}.
Pokud to pro vás nestačí, pak je tu vynikající kniha Just My Type \cite{Garfield2010}. Autor nám říká, proč lidé nemají rádi Papyrus, Trajan Capitals a mnohem víc.

\subsection{Projev v~umění}
Nová média a související nové typy publikací, které se objevily na konci 19. a 20. století, vyvolaly nové otázky týkající se úlohy typografie ve vydavatelství. Víc v~\cite{Media_Research2016}. 

Česká spisovatelka v~knize  \cite{Vacovska2016} vypráví, jak vypadala typografie filmových titulků československé produkce.


Pokud vás zajímá typografie ve webdesignu, můžete si o~ní přečíst zde \cite{Young_scientist2020}.

\subsection{Copywriteři}

Vědomosti jsou nutné a někdy i nudné. Praktické zkušenosti jsou ale k~nezaplacení. Mini rozhovor \cite{Celevicek2018} se 3 známých copywriterů, které pracují s~typografií v~obsahu. 

\subsection{Zajímavé informace}

V~časopise Font \cite{Font2017} si můžete přečíst o~analýze zahraničních a domácích časopisů, zejména jejich obálek, třeba Cosmopolitan.





\newpage

\bibliographystyle{czechiso}
\renewcommand{\refname}{Použitá Literatura}
\bibliography{proj4}

\end{document}
