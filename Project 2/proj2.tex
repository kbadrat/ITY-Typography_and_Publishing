\documentclass[a4paper, 11pt, twocolumn]{article}

\usepackage[left=1.5cm,top=2.5cm,text={18cm, 25cm}]{geometry}
\usepackage[czech]{babel}
\usepackage[utf8]{inputenc}
\usepackage{times}
\usepackage{amssymb}
\usepackage{amsthm}
\usepackage{amsmath}
\newtheorem{theorem}{Definice}
\newtheorem{sentence}{Věta}
\usepackage{bm}
\usepackage[IL2]{fontenc}

\begin{document}

\begin{titlepage}
\begin{center}

\Huge \textsc{Fakulta informačních technologií\\
Vysoké učení technické v Brně} \\
\vspace{\stretch{0.3}}
\LARGE Typografie a publikování \,--\, 2. projekt \\
Sazba dokumentů a matematických výrazů \\
\vspace{\stretch{0.4}}

\end{center}

{\Large 2021 \hfill
Kovalets Vladyslav (xkoval21)}

\end{titlepage}

\newpage

\section*{Úvod}

V~této úloze si vyzkoušíme sazbu titulní strany, matematic\-kých vzorců, prostředí a dalších textových struktur obvyklých pro technicky zaměřené texty (například rovnice (\ref{equation_1})
nebo Definice \ref{proof_1} na straně \pageref{proof_1}). Rovněž si vyzkoušíme pou\-žívání odkazů \verb|\ref| a \verb|\pageref|.

Na titulní straně je využito sázení nadpisu podle optického středu s~využitím zlatého řezu. Tento postup byl
probírán na přednášce. Dále je použito odřádkování se
zadanou relativní velikostí 0.4 em a 0.3 em.

V~případě, že budete potřebovat vyjádřit matematickou
konstrukci nebo symbol a nebude se Vám dařit jej nalézt
v~samotném \LaTeX u, doporučuji prostudovat možnosti
balíku maker \AmS-\LaTeX.

\section{Matematický text}

Nejprve se podíváme na sázení matematických symbolů
a výrazů v~plynulém textu včetně sazby definic a vět s~využitím balíku \verb|amsthm|. Rovněž použijeme poznámku pod
čarou s~použitím příkazu \verb|\footnote|. Někdy je vhodné
použít konstrukci \verb|\mbox{}|, která říká, že text nemá být
zalomen.


\begin{theorem}
\label{proof_1}
\emph{Rozšířený zásobníkový automat (RZA)} je definován jako sedmice tvaru
 $A = (Q, \Sigma, \Gamma, \delta, q_{0}, Z_{0}, F),$
kde:


\begin{itemize}
\item[$\bullet$] $Q$ je konečná množina \emph{vnitřních (řídicích) stavů},
\item[$\bullet$] $\Sigma$ je konečná \emph{vstupní abeceda},
\item[$\bullet$] $\Gamma$ je konečná \emph{zásobníková abeceda},
\item[$\bullet$] $\delta$ je \emph{přechodová funkce} $Q \times(\Sigma \cup\{\epsilon\}) \times \Gamma^{*} \rightarrow 2^{Q \times \Gamma^{*}}$,
\item[$\bullet$] $q_{0} \in Q$ je \emph{počáteční stav},$Z_{0} \in \Gamma$ je \emph{startovací symbol
zásobníku} a $\ F \subseteq Q$ je množina \emph{koncových stavů}.
\end{itemize}

\emph{Nechť $P=\left(Q, \Sigma, \Gamma, \delta, q_{0}, Z_{0}, F\right)$ je rozšířený zásobníkový automat.} Konfigurací \emph{nazveme trojici $(q, w, \alpha) \in
Q \times \Sigma^{*} \times \Gamma^{*}$, kde $q$ je aktuální stav vnitřního řízení,
$w$ je dosud nezpracovaná část vstupního řetězce a $\alpha=
Z_{i_{1}} Z_{i_{2}} \ldots Z_{i_{k}}$ je obsah zásobníku\footnote{$Z_{i_{1}}$ je vrchol zásobníku}}.

\end{theorem}

\subsection{Podsekce obsahující větu a odkaz}
\begin{theorem}
\label{proof_2}
\emph{Řetězec $w$ nad abecedou $\Sigma$ je přijat RZA}
A jestliže $\left(q_{0}, w, Z_{0}\right)\underset{A}{\overset{*}{\vdash}} \left(q_{F}, \epsilon, \gamma\right)$ pro nějaké $\gamma \in \Gamma^{*}$ a $ 
q_{F} \in F$. Množinu $L(A)=\{w \mid w$ je přijat RZA  A$\} \subseteq \\ 
\Sigma^{*}$ nazýváme \emph{jazyk přijímaný RZA} A.
\end{theorem}

Nyní si vyzkoušíme sazbu vět a důkazů opět s použitím
balíku \verb|amsthm|.

\begin{sentence}
Třída jazyků, které jsou přijímány ZA, odpovídá
\emph{bezkontextovým jazykům}.
\end{sentence}
\begin{proof}
V důkaze vyjdeme z Definice \ref{proof_1} a \ref{proof_2}.
\end{proof}

\section{Rovnice a odkazy}

Složitější matematické formulace sázíme mimo plynulý
text. Lze umístit několik výrazů na jeden řádek, ale pak je
třeba tyto vhodně oddělit, například příkazem \verb|\quad|.

$$\sqrt[i]{x_{i}^{3}} \quad \text {kde} \ x_{i} \ \text {je} \ i \text {-té sudé číslo splňující} \quad x_{i}^{x_{i}^{i^{2}}+2} \leq y_{i}^{x_{i}^{4}}$$

V~rovnici (\ref{equation_1}) jsou využity tři typy závorek s různou
explicitně definovanou velikostí.
\begin{eqnarray}
\label{equation_1} x&=&\left [\Big\{\big[a+b\big] * c\Big\}^{d} \oplus 2\right]^{3 / 2}\\
y&=&\lim _{x \to \infty} \frac{\frac{1}{\log _{10} x}}{\sin ^{2} x+\cos ^{2} x} \nonumber
\end{eqnarray}





V~této větě vidíme, jak vypadá implicitní vysázení limity lim${ }_{n \to \infty} f(n)$ v~normálním odstavci textu. Podobně
je to i s~dalšími symboly jako $\prod_{i=1}^{n} 2^{i}$ či $\bigcap_{A \in \mathcal{B}} A$. V~případě vzorců $ \lim\limits_{n\to\infty} f(n)$ a $
\prod\limits_{i=1}^{n} 2^{i}$ jsme si vynutili méně
úspornou sazbu příkazem \verb|\limits|. 

\begin{equation}
\int_{b}^{a} g(x) \mathrm{d} x\quad=\quad-\int\limits^b_a f(x) \mathrm{d} x
\end{equation}

\section{Matice}
Pro sázení matic se velmi často používá prostředí \verb|array|
a závorky (\verb|\left|, \verb|\right|).

$$
\left(\begin{array}{ccc}
a-b & \widehat{\xi+\omega} & \pi \\
\Vec{\mathbf{a}} & \overleftrightarrow{A C} & \hat{\beta}
\end{array}\right)=1 \Longleftrightarrow \mathcal{Q}=\mathbb{R}
$$

$$
\mathbf{A}=\left\|\begin{array}{cccc}
a_{11} & a_{12} & \ldots & a_{1 n} \\
a_{21} & a_{22} & \ldots & a_{2 n} \\
\vdots & \vdots & \ddots & \vdots \\
a_{m 1} & a_{m 2} & \ldots & a_{m n}
\end{array}\right\|=\left|\begin{array}{cc}
t & u \\
v & w
\end{array}\right| = tw\!-\!uv
$$

Prostředí \verb|array| lze úspěšně využít i jinde.

$$
\begin{pmatrix}
n\\k
\end{pmatrix}
=\left\{\begin{array}{cl}
0 & \text { pro } k<0 \text { nebo } k>n \\
\frac{n!}{k !(n - k) !} & \text { pro } 0 \leq k \leq n.
\end{array}\right.
$$

\end{document}
